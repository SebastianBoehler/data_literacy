%%%%%%%% DATA LITERACY 2025 LATEX PROJECT TEMPLATE FILE %%%%%%%%%%%%%%%%%
%%% Based on the 2025 ICML template, available at https://icml.cc/Conferences/2025/AuthorInstructions %%%

\documentclass{article}

% Recommended, but optional, packages for figures and better typesetting:
\usepackage{microtype}
\usepackage{graphicx}
\usepackage{subfigure}
\usepackage{booktabs} % for professional tables

\usepackage{tikz}
% Corporate Design of the University of Tübingen
% Primary Colors
\definecolor{TUred}{RGB}{165,30,55}
\definecolor{TUgold}{RGB}{180,160,105}
\definecolor{TUdark}{RGB}{50,65,75}
\definecolor{TUgray}{RGB}{175,179,183}

% Secondary Colors
\definecolor{TUdarkblue}{RGB}{65,90,140}
\definecolor{TUblue}{RGB}{0,105,170}
\definecolor{TUlightblue}{RGB}{80,170,200}
\definecolor{TUlightgreen}{RGB}{130,185,160}
\definecolor{TUgreen}{RGB}{125,165,75}
\definecolor{TUdarkgreen}{RGB}{50,110,30}
\definecolor{TUocre}{RGB}{200,80,60}
\definecolor{TUviolet}{RGB}{175,110,150}
\definecolor{TUmauve}{RGB}{180,160,150}
\definecolor{TUbeige}{RGB}{215,180,105}
\definecolor{TUorange}{RGB}{210,150,0}
\definecolor{TUbrown}{RGB}{145,105,70}

% hyperref makes hyperlinks in the resulting PDF.
\usepackage{hyperref}

% Attempt to make hyperref and algorithmic work together better:
\newcommand{\theHalgorithm}{\arabic{algorithm}}

\usepackage[accepted]{icml2025}

% For theorems and such
\usepackage{amsmath}
\usepackage{amssymb}
\usepackage{mathtools}
\usepackage{amsthm}

% if you use cleveref..
\usepackage[capitalize,noabbrev]{cleveref}

% Todonotes is useful during development
\usepackage[textsize=tiny]{todonotes}

% The \icmltitle you define below is probably too long as a header.
\icmltitlerunning{Analysis of Bus Delays in Tübingen}

\begin{document}

\twocolumn[
\icmltitle{Are Tübingen Buses Really Always Late?\\An Analysis of Public Transit Delays}

\begin{icmlauthorlist}
\icmlauthor{Firstname1 Lastname1}{equal,first}
\icmlauthor{Firstname2 Lastname2}{equal,second}
\icmlauthor{Firstname3 Lastname3}{equal,third}
\icmlauthor{Firstname4 Lastname4}{equal,fourth}
\icmlauthor{Alexandra Keller}{equal,fifth}
\end{icmlauthorlist}

\icmlaffiliation{first}{Matrikelnummer 12345678, MSc Machine Learning}
\icmlaffiliation{second}{Matrikelnummer 12345678, MSc Computer Science}
\icmlaffiliation{third}{Matrikelnummer 12345678, MSc Media Informatics}
\icmlaffiliation{fourth}{Matrikelnummer 12345678, MSc Medical Informatics}
\icmlaffiliation{fifth}{Matrikelnummer 12345678, MSc Cognitive Science}

\icmlcorrespondingauthor{F1}{first1.last1@uni-tuebingen.de} 
\icmlcorrespondingauthor{F2}{first2.last2@uni-tuebingen.de}
\icmlcorrespondingauthor{F3}{first3.last3@uni-tuebingen.de}
\icmlcorrespondingauthor{F4}{first4.last4@uni-tuebingen.de}
\icmlcorrespondingauthor{AK}{alexandra.keller@uni-tuebingen.de}

\icmlkeywords{Public Transport, Delay Analysis, Data Literacy}

\vskip 0.3in
]

\printAffiliationsAndNotice{\icmlEqualContribution}

\begin{abstract}
Many cities, including Tübingen, rely on an efficiently functioning bus network for public transport. By 2025, public opinion was that buses were frequently delayed and suboptimal, which put a strain on students' and clinicians' ability to get to work. To test these perceptions, we collected and analyzed bus arrival information provided via the TRIAS public transport interface, operated by the regional transit authority (via the MobiData BW/efa-bw service). We analyzed the delays by bus line, bus stop, time of day and weather conditions. Our analysis reveals that 81\% of departures are on time, with a mean delay of only 0.61 minutes---substantially better than public perception suggests. However, evening rush hours show elevated delays, with 10\% of passengers experiencing delays approaching 5 minutes.
\end{abstract}

\section{Introduction}\label{sec:intro}

The daily lives of many citizens are dependent on a properly functioning public transport system. In Tübingen, as in many other cities, the public transport is realised by a city-wide bus system. However, citizens perceive the provided buses as suboptimal. In 2025, students of Tübingen started an initiative, describing delays and overly crowded buses, putting a strain on students' and clinicians' ability to get to work. Complaints were going as far as saying ``I don't go to the lectures anymore, due to the bus situation'' \citep{busstattfrust2025}. Tübingen aims at becoming a sustainable city, yet announced plans to reduce the frequency at which buses arrive even further.

Within our report we aim to analyse, whether the buses are indeed as delayed as subjectively experienced and where and which buses are primarily affected. Looking into the officially tracked delays by the city, allows us to back up or counterfeit the subjective perception with statistical and graphical analysis. Furthermore, it might reveal specific points for improvement, regarding the exact lines, the day times, and regions affected by delays. Improving the bus system as a sustainable, eco-friendly and socially just transport option is of relevance for Tübingen and its citizens.

\section{Data and Methods}\label{sec:methods}

For our study, we analyzed bus arrival information provided through the TRIAS public transport interface operated by the regional transit authority (via the MobiData BW / efa-bw service). We implemented a scheduled cloud function that queries the TRIAS API at hourly intervals and retrieves real-time departure information for bus stops within a fixed spatial radius around the city of Tübingen. Each query produced a snapshot of estimated and scheduled arrivals for multiple upcoming departures, allowing us to observe how estimated delays vary across bus lines, locations, and times of day.

In addition to stop-level departure data, we retrieved trip-level schedule information describing complete bus journeys. For each journey, the TRIAS API returns the full sequence of stops along the route, including stops the bus has already visited and stops it will visit in the future. To ensure our analysis uses only realized delays rather than predictions, we filter the trip data to include only stops labelled as previously visited.

For each observed stop event, we record the bus line, destination, stop identifier, scheduled arrival time, and estimated arrival time. From the scheduled and estimated arrival timestamps, we compute a delay variable defined as the difference between the estimated and scheduled arrival time, measured in minutes.

In the available data, estimated arrival times are never earlier than the scheduled times, such that negative delay values do not occur. This should not be interpreted as buses never arriving early, but rather reflects a limitation of the real-time data, which does not explicitly report early arrivals. Consequently, a delay value of zero represents arrivals that are either on time or early relative to the schedule.

Data was collected continuously over a period spanning from mid-November 2025 to mid-January 2026. Implausible delay values exceeding 90 minutes were removed as data errors.

\begin{figure}[ht]
\vskip 0.2in
\begin{center}
\centerline{\includegraphics[width=\columnwidth]{images/fig1_eda.pdf}}
\caption{Exploratory data analysis of Tübingen bus delays. (A) Distribution of delays showing right-skewed pattern with spike at zero. (B) Average delay by hour of day with 95\% CI. (C) Top 10 busiest stops. (D) Average delay by weather condition.}
\label{fig:eda}
\end{center}
\vskip -0.2in
\end{figure}

\section{Results}\label{sec:results}

\subsection{Service Reliability Overview}

Analysis of bus departures across 68 lines reveals that Tübingen's bus network performs better than public perception suggests. The overall mean delay is 0.61 minutes (95\% CI: [0.59, 0.62]), with a median of 0 minutes, indicating that the majority of departures are on time or early. Approximately 81\% of departures arrive on time ($\leq$0 minutes delay). Among the 19\% of buses that are late, the average delay is 3.20 minutes.

These metrics are comparable to or exceed those reported by other German cities. However, direct comparison is complicated by differing threshold definitions:
\begin{itemize}
    \item \textbf{Munich (MVG)}: 69.5\% on-time using a $\leq$3 minute threshold
    \item \textbf{Bremen (VBN)}: 89.0\% on-time at $\leq$5 minute threshold
    \item \textbf{Duisburg (DVG)}: 69.3\% on-time at $\leq$3 minutes
\end{itemize}

Unlike these aggregate reports, our dataset captures individual departures with real-time estimates, enabling analysis of temporal patterns, spatial variation, and distributional characteristics not visible in summary statistics.

The distribution of delays is strongly right-skewed (\Cref{fig:eda}A), with a pronounced spike at zero and a long tail of late arrivals. This asymmetry means that while most passengers experience punctual service, a minority encounter substantial delays, which may disproportionately shape negative perceptions.

\subsection{Temporal Patterns}

Time-of-day analysis reveals clear peak-hour effects (\Cref{fig:eda}B). The evening peak (16--19h) shows the highest average delay at 1.00 minutes, with hour 17 reaching 1.34 minutes---the daily maximum. The 90th percentile delay peaks at approximately 4.7 minutes during hour 17, compared to near-zero during early morning hours. This means that while most evening commuters experience modest delays, 10\% face delays approaching 5 minutes.

Morning rush hours (6--9h) show a more moderate average of 0.48 minutes. Off-peak periods exhibit substantially lower delays: midday (10--15h) averages 0.52 minutes, and late evening (20--23h) averages just 0.28 minutes.

Weekend delays are systematically lower than weekday delays (mean 0.31 vs 0.68 minutes, non-overlapping 95\% CIs). While we lack direct traffic data, this pattern is consistent with typical weekend conditions in urban areas.

\begin{figure}[ht]
\vskip 0.2in
\begin{center}
\centerline{\includegraphics[width=\columnwidth]{images/fig2_schedule.pdf}}
\caption{Average delay before vs. after the December 14, 2025 schedule change, with 95\% confidence intervals.}
\label{fig:schedule}
\end{center}
\vskip -0.2in
\end{figure}

\subsection{Impact of Schedule Change}

On December 14, 2025, the city implemented schedule adjustments aimed at cost reduction and operational efficiency. Comparing the periods before and after this change reveals a substantial reduction in mean delay from 0.85 minutes to 0.38 minutes---a reduction of over 50\% (\Cref{fig:schedule}).

This improvement should be interpreted cautiously. The post-change period coincides with the Christmas holiday season, during which reduced traffic and lower ridership may independently contribute to improved punctuality.

\subsection{Weather and Line-Specific Effects}

Contrary to expectations, weather conditions show minimal association with delays (\Cref{fig:eda}D). Dry conditions dominate the dataset, while adverse conditions occur infrequently. Surprisingly, adverse weather conditions (snow, hail, sleet) show \emph{lower} average delays than dry conditions, possibly reflecting reduced traffic volumes.

Delay patterns vary substantially across bus lines. Line 18 exhibits the highest mean delay (2.11 min)---notably higher than all other lines. Lines 16 (1.05 min) and 19 (1.01 min) also show elevated delays, while Lines 1, 9, and 11 perform near or below the system average.

\section{Discussion \& Conclusion}\label{sec:conclusion}

Our analysis reveals that Tübingen's bus system performs substantially better than public perception suggests, with 81\% of departures on time and a mean delay of only 0.61 minutes. However, the right-skewed delay distribution means that a minority of passengers experience substantial delays, which may disproportionately shape negative perceptions.

The citizen initiative highlighted concerns about delays to the University and Clinics during rush hours. Our data confirms elevated delays during evening rush hours (16--19h), with 10\% of passengers facing delays approaching 5 minutes. Line 18, specifically mentioned in complaints, shows the highest mean delay among all lines.

\textbf{Limitations:} The TRIAS API does not report early arrivals; delay=0 may include early buses. The post-schedule-change period overlaps with Christmas holidays, confounding interpretation. Crowding and cancelled buses are not captured in our data.

\section*{Contribution Statement}
[To be filled in]

\bibliography{bibliography}
\bibliographystyle{icml2025}

\end{document}
